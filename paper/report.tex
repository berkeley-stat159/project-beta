\documentclass[11pt]{article}

\usepackage[margin=0.75in]{geometry}

\title{Modeling of Semantic Representation in the brain using fMRI response}
\author{
  Sinha, Rishi\\
  \texttt{rishizsinha}
  \and
  Mo, Cindy\\
  \texttt{cxmo}
  \and
  Agrawal, Raj\\
  \texttt{raj4}
  \and
  Wang, Yuan (Aria)\\
  \texttt{ariaaay}
  \and
  Dong, Yucheng (Steve)\\
  \texttt{yuchengdong}
}

\bibliographystyle{siam}

\begin{document}
\maketitle

\abstract{You should have a short abstract.}
We attempt to find correlations between brain activity through fMRI scans and emotions 
or actions occuring in the movie. The data provided includes a detailed transcript of 
the movie divided by scenes. This process involves NLP (Natural Language Processing) of 
the script using text-sentiment/n-gram analysis of the script to determine the "emotional 
distribution" of each scene. After training a linear regression model based on multiple 
scsene/ fMRI pairs, we will predict how people are feeling or actions that are occuring 
based on fMRI scans and the active areas of the brain during that period.  

\section{Introduction}

\section{Data}
 The original audio description was in German, so we first used Google Translate 
 to convert from German to English. Due to grammatical differences, we decided to
 only keep nouns and verbs, and discarded the adjectives and other words including
 stopwords. Princeton University provides a list of stopwords for text preprocessing.
 Stopwords are the most commonly used natural language words in the English, but have 
 very little meaning. Examples of stop words include "and", "to", and "him". We saved 
 the translated text into a CSV file and parsed in Python. 

\section{Methods}
\section{Results}
\section{Discussion}


\bibliography{project}

\end{document}
